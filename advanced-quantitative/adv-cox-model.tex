\documentclass[9pt,xcolor=dvipsnames,aspectratio=169]{beamer}
\usepackage[utf8]{inputenc}
\usepackage{amsmath,amssymb,graphicx,tikz,pgfplots,booktabs,siunitx}
\usetikzlibrary{arrows,shapes,decorations.pathmorphing,decorations.pathreplacing,decorations.snapping,fit,positioning,calc,intersections,shapes.geometric,backgrounds}
\usetheme[numbering=fraction,titleformat=smallcaps,sectionpage=progressbar]{metropolis}
\usepackage[style=authoryear]{biblatex}
\addbibresource{references.bib}
\setbeamertemplate{bibliography item}[text]
\graphicspath{{../assets/}}
\DeclareMathOperator{\e}{e}
\title{\Large SDS6210: Informatics for Health\\[0.3em]\small Advanced Quantitative: Cox Proportional Hazards Model}
\author{\textbf{Cavin Otieno}}
\institute{MSc Public Health Data Science\\Department of Health Informatics}
\date{\today}
\begin{document}
\begin{frame}[noframenumbering,plain]
    \maketitleslide
\end{frame}
\section{Survival Analysis Fundamentals}
\begin{frame}{Hazard Function: Definition and Interpretation}
The hazard function $h(t|\mathbf{x})$ represents the instantaneous rate of event occurrence at time $t$, conditional on survival up to time $t$. It is a fundamental concept in survival analysis that characterizes the instantaneous risk of the event.

Mathematically:
\[
h(t|\mathbf{x}) = \lim_{\Delta t \to 0} \frac{P(t \leq T < t + \Delta t | T \geq t)}{\Delta t}
\]

In terms of the probability density function $f(t)$ and survival function $S(t)$:
\[
h(t) = \frac{f(t)}{S(t)} = -\frac{d}{dt} \log S(t)
\]

The cumulative hazard function:
\[
H(t) = \int_{0}^{t} h(u) \, du = -\log S(t)
\]

And the survival function:
\[
S(t) = \exp\left(-H(t)\right) = \exp\left(-\int_{0}^{t} h(u) \, du\right)
\]

Common parametric forms:
\begin{center}
\scalebox{0.75}{
\begin{tabular}{@{}llp{5cm}@{}}
\toprule \textbf{Distribution} & \textbf{Hazard Function} & \textbf{Interpretation} \\
\midrowcolor
Exponential | $h(t) = \lambda$ | Constant hazard (memoryless) |
Weibull | $h(t) = \lambda \gamma (\lambda t)^{\gamma-1}$ | Increasing/decreasing hazard |
Gompertz | $h(t) = \lambda e^{\gamma t}$ | Arbitrary hazard shape \\
\bottomrule
\end{tabular}}
\end{center}
\end{frame}
\begin{frame}{The Cox Proportional Hazards Model}
The Cox proportional hazards model, developed by Sir David Cox in 1972, is the most widely used semi-parametric model for survival data. It allows estimation of covariate effects on the hazard rate without specifying the baseline hazard function.

The model specification:
\[
h(t|\mathbf{x}) = h_0(t) \exp(\beta_1 x_1 + \beta_2 x_2 + \cdots + \beta_p x_p)
\]

Equivalently:
\[
\log h(t|\mathbf{x}) = \log h_0(t) + \beta_1 x_1 + \cdots + \beta_p x_p
\]

Key features:
\begin{center}
\scalebox{0.75}{
\begin{tabular}{@{}llp{5cm}@{}}
\toprule \textbf{Component} & \textbf{Interpretation} & \textbf{Properties} \\
\midrowcolor
$h_0(t)$ | Baseline hazard (when all $x_j = 0$) | Non-parametric, cancels out |
$\beta_j$ | Log-hazard ratio for unit increase in $x_j$ | Estimated from data |
$e^{\beta_j}$ | Hazard ratio comparing levels of $x_j$ | Exponentiated coefficient |
\bottomrule
\end{tabular}}
\end{center}

The proportional hazards assumption:
\[
\frac{h(t|\mathbf{x}_1)}{h(t|\mathbf{x}_2)} = \exp\left[\boldsymbol{\beta}'(\mathbf{x}_1 - \mathbf{x}_2)\right]
\]

This ratio is constant over time, independent of $t$.
\end{frame}
\begin{frame}{Hazard Ratio Interpretation}
The exponentiated coefficients have direct interpretation as hazard ratios:

For a binary covariate (treatment vs. control):
\[
\text{HR} = \frac{h(t|\text{treatment})}{h(t|\text{control})} = \exp(\beta_{\text{treatment}})
\]

Interpretation:
\begin{center}
\scalebox{0.75}{
\begin{tabular}{@{}llp{5cm}@{}}
\toprule \textbf{HR Value} & \textbf{Interpretation} & \textbf{Example} \\
\midrowcolor
HR = 1.0 | No difference in hazard | Placebo vs. placebo |
HR < 1.0 | Treatment reduces hazard | New drug (HR = 0.70) |
HR > 1.0 | Treatment increases hazard | Harmful exposure (HR = 1.5) \\
\bottomrule
\end{tabular}}
\end{center}

For a continuous covariate (Age):
\[
\text{HR} = \exp(\beta_{\text{Age}}) \quad \text{per unit increase in Age}
\]

For a 10-year age difference:
\[
\text{HR}_{10\text{ years}} = \exp(10 \times \beta_{\text{Age}}) = \left[\exp(\beta_{\text{Age}})\right]^{10}
\]

Example: If $\beta_{\text{Age}} = 0.05$, then HR per year = 1.05, and HR per 10 years = 1.63. A 60-year-old has 63\% higher hazard than a 50-year-old.
\end{frame}
\begin{frame}{Partial Likelihood Estimation}
Cox's key innovation was recognizing that the baseline hazard $h_0(t)$ need not be specified. The partial likelihood depends only on the covariate effects:

For subjects who experience events at distinct times $t_1 < t_2 < \ldots < t_K$:
\[
L_P(\boldsymbol{\beta}) = \prod_{k=1}^{K} \frac{\exp\left[\boldsymbol{\beta}' \mathbf{x}_{(k)}\right]}{\sum_{j \in R(t_k)} \exp\left[\boldsymbol{\beta}' \mathbf{x}_j\right]}
\]

where $\mathbf{x}_{(k)}$ is the covariate vector for the subject failing at $t_k$, and $R(t_k)$ is the risk set at $t_k$.

The log-partial likelihood:
\[
\ell_P(\boldsymbol{\beta}) = \sum_{k=1}^{K} \left[\boldsymbol{\beta}' \mathbf{x}_{(k)} - \log \sum_{j \in R(t_k)} \exp\left(\boldsymbol{\beta}' \mathbf{x}_j\right)\right]
\]

The score function:
\[
U(\boldsymbol{\beta}) = \frac{\partial \ell_P}{\partial \boldsymbol{\beta}} = \sum_{k=1}^{K} \left[\mathbf{x}_{(k)} - \bar{\mathbf{x}}_w(t_k)\right]
\]
where $\bar{\mathbf{x}}_w(t_k)$ is the weighted mean of covariates in the risk set.
\end{frame}
\begin{frame}{Assumptions of the Cox Model}
The Cox proportional hazards model relies on several assumptions:

\begin{center}
\scalebox{0.75}{
\begin{tabular}{@{}llp{5cm}@{}}
\toprule \textbf{Assumption} & \textbf{Description} & \textbf{Testing Method} \\
\midrowcolor
Proportional hazards | HR constant over time & Schoenfeld residuals, time-by-covariate interaction |
Linear form | Log-hazard linear in covariates | Residual plots, component-plus-residual plots |
Independence | Independent event times | Study design, clustering analysis |
Correct model | All important predictors included & Subject matter knowledge, model building \\
\bottomrule
\end{tabular}}
\end{center}

Testing proportional hazards assumption using Schoenfeld residuals:
\[
\text{Test statistic} = \sum_{k=1}^{K} \hat{\mathbf{s}}_k \otimes t_k
\]
where $\hat{\mathbf{s}}_k$ are the Schoenfeld residuals at time $t_k$ and $t_k$ is the ranked event time.

If the correlation between scaled Schoenfeld residuals and time is significant, the proportional hazards assumption is violated.

Addressing proportional hazards violations:
\begin{center}
\scalebox{0.75}{
\begin{tabular}{@{}llp{5cm}@{}}
\toprule \textbf{Approach} & \textbf{Method} & \textbf{Application} \\
\midrowcolor
Stratification | Separate baseline hazards by stratum | Different effects across strata |
Time-dependent covariates | Include $x \times \log(t)$ interaction | Changing effects over time |
Parametric models | Specify baseline hazard form | When PH clearly violated \\
\bottomrule
\end{tabular}}
\end{center}
\end{frame}
\section{Application Examples}
\begin{frame}{Example: HIV Mortality Study in Kenya}
Consider a cohort study of 1,000 HIV-positive adults initiating ART:

Model specification:
\[
h(t|\mathbf{x}) = h_0(t) \exp\left(\beta_1 \text{CD4} + \beta_2 \text{Age} + \beta_3 \text{Stage} + \beta_4 \text{Adherence}\right)
\]

Estimated coefficients:
\begin{center}
\scalebox{0.75}{
\begin{tabular}{@{}lllp{4cm}@{}}
\toprule \textbf{Covariate} & \textbf{HR} & \textbf{95\% CI} & \textbf{Interpretation} \\
\midrowcolor
CD4 count (per 100 cells) | 0.72 | (0.65, 0.80) | Higher CD4, lower mortality |
Age (per 10 years) | 1.25 | (1.10, 1.42) | Older age, higher mortality |
WHO Stage III/IV vs I/II | 1.85 | (1.40, 2.44) | Advanced disease, higher mortality |
Adherence ≥95\% vs <95\% | 0.45 | (0.32, 0.63) | Better adherence, lower mortality \\
\bottomrule
\end{tabular}}
\end{center}

Interpretation example:
\[
\text{HR}_{\text{CD4}} = 0.72 \quad \text{per 100 cells increase}
\]
\[
\text{HR}_{200 \text{ vs } 100} = \exp(2 \times \log(0.72)) = 0.72^2 = 0.52
\]
Patients with CD4 count 200 cells/μL have 48\% lower mortality hazard than those with 100 cells/μL, adjusting for other factors.
\end{frame}
\begin{frame}{Stratified Cox Model}
When the proportional hazards assumption fails for a categorical variable, the stratified Cox model allows separate baseline hazards for each stratum:

\[
h_s(t|\mathbf{x}) = h_{0s}(t) \exp\left(\boldsymbol{\beta}' \mathbf{x}\right), \quad s = 1, \ldots, S
\]

The partial likelihood sums over events within each stratum:
\[
L_P(\boldsymbol{\beta}) = \prod_{s=1}^{S} \prod_{k \in \text{events in } s} \frac{\exp\left[\boldsymbol{\beta}' \mathbf{x}_{(k)}\right]}{\sum_{j \in R_s(t_k)} \exp\left[\boldsymbol{\beta}' \mathbf{x}_j\right]}
\]

Example: Stratifying by facility type in a multi-site study
\[
h_{\text{teaching}}(t|\mathbf{x}) = h_{0,\text{teaching}}(t) \exp(\beta_1 \text{CD4} + \beta_2 \text{Age})
\]
\[
h_{\text{non-teaching}}(t|\mathbf{x}) = h_{0,\text{non-teaching}}(t) \exp(\beta_1 \text{CD4} + \beta_2 \text{Age})
\]

The coefficients $\beta_1$ and $\beta_2$ are assumed constant across strata, but baseline hazards can differ, accommodating different baseline risks across facility types.
\end{frame}
\begin{frame}{Time-Dependent Covariates}
When covariate effects change over time, time-dependent covariates can be incorporated:

Model with time-dependent effect:
\[
h(t|\mathbf{x}(t)) = h_0(t) \exp\left[\beta_1 x_1 + \beta_2 x_1 \log(t)\right]
\]

The hazard ratio for $x_1$ at time $t$:
\[
\text{HR}(t) = \exp\left[\beta_1 + \beta_2 \log(t)\right] = t^{\beta_2} e^{\beta_1}
\]

Example: Effect of treatment that diminishes over time
\[
\text{HR}(t) = \exp\left[0.50 - 0.15 \log(t)\right] = t^{-0.15} e^{0.50}
\]

\begin{center}
\begin{tikzpicture}[scale=0.8]
\begin{axis}[
    axis lines=middle,
    xlabel={Time (months)},
    ylabel={Hazard Ratio},
    xmin=0, xmax=24,
    ymin=0.5, ymax=2.5,
    legend pos=north east,
    width=0.7\textwidth,
    height=0.5\textwidth,
    domain=1:24
]
\addplot[blue,thick,samples=200] {exp(0.50 - 0.15*ln(x))};
\addlegendentry{Diminishing treatment effect}
\addplot[red,dashed,thick] coordinates {(1,1.65) (24,1.65)};
\addlegendentry{Constant effect (no time dep)}
\end{axis}
\end{tikzpicture}
\end{center}
\end{frame}
\section{LMIC Context: Sub-Saharan Africa}
\begin{frame}{Cox Model Applications in African Health Research}
The Cox proportional hazards model is widely used in African health research:

\begin{center}
\scalebox{0.75}{
\begin{tabular}{@{}llp{5cm}@{}}
\toprule \textbf{Application} & \textbf{Setting} & \textbf{Research Question} \\
\midrowcolor
HIV treatment outcomes | Kenya, South Africa | Predictors of mortality on ART |
Malaria chemoprevention | Malawi, Senegal | Duration of protection from seasonal malaria chemoprevention |
Maternal mortality | Multiple African countries | Risk factors for maternal death |
Child survival | Ghana, Ethiopia | Impact of interventions on under-5 mortality \\
\bottomrule
\end{tabular}}
\end{center}

Example: AMPATH HIV cohort analysis (Kenya)

\[
h(t|\mathbf{x}) = h_0(t) \exp\left(\beta_1 \text{CD4}_0 + \beta_2 \text{Age} + \beta_3 \text{Gender} + \beta_4 \text{Adherence}\right)
\]

Key findings from Cox model analysis:
\begin{center}
\scalebox{0.75}{
\begin{tabular}{@{}llp{5cm}@{}}
\toprule \textbf{Factor} & \textbf{HR} & \textbf{Implication} \\
\midrowcolor
Baseline CD4 (<200 vs >350) | 2.4 | Early mortality risk requires intensive support |
Adherence support program | 0.6 | Interventions reducing mortality |
Transfer to rural facility | 1.3 | Retention challenges in rural settings \\
\bottomrule
\end{tabular}}
\end{center}
\end{frame}
\begin{frame}{Challenges for Cox Model in LMICs}
Implementing Cox model analysis in resource-constrained settings:

\begin{center}
\scalebox{0.75}{
\begin{tabular}{@{}llp{5cm}@{}}
\toprule \textbf{Challenge} & \textbf{Impact} & \textbf{Mitigation Strategy} \\
\midrowcolor
Censoring patterns | Informative censoring bias | Sensitivity analyses, IPCW |
Loss to follow-up | Reduced power, selection bias | Tracing programs, mobile tracking |
Competing risks | Death from other causes | Cause-specific hazards, subdistribution hazards |
Software availability | Limited statistical software | R (survival), Stata (stcox), Python (lifelines) \\
\bottomrule
\end{tabular}}
\end{center}

Competing risks framework for HIV patients:
\[
h_j(t) = \lim_{\Delta t \to 0} \frac{P(t \leq T < t + \Delta t, D=j | T \geq t)}{\Delta t}
\]
where $D$ is the cause of failure (AIDS vs. other death).

Summary of approaches:
\begin{center}
\scalebox{0.75}{
\begin{tabular}{@{}llp{5cm}@{}}
\toprule \textbf{Approach} & \textbf{Research Question} & \textbf{LMIC Application} \\
\midrowcolor
Cause-specific hazard | Factors increasing specific cause of death | Identifying preventable causes \\
Subdistribution hazard | Probability of failure from specific cause | Cumulative incidence functions \\
\bottomrule
\end{tabular}}
\end{center}
\end{frame}
\section{Summary}
\begin{frame}{Key Takeaways}
\begin{enumerate}
1. The hazard function $h(t) = f(t)/S(t)$ represents the instantaneous event rate conditional on survival to time $t$.
\end{enumerate}

\begin{enumerate}
\resume{enumerate}
2. The Cox model $h(t|\mathbf{x}) = h_0(t) \exp(\boldsymbol{\beta}'\mathbf{x})$ separates baseline hazard from covariate effects.
\end{enumerate}

\begin{enumerate}
\resume{enumerate}
3. Exponentiated coefficients are hazard ratios: $\exp(\beta_j)$ represents the multiplicative effect on hazard.
\end{enumerate}

\begin{enumerate}
\resume{enumerate}
4. The partial likelihood enables estimation without specifying the baseline hazard function.
\end{enumerate}

\begin{enumerate}
\resume{enumerate}
5. Key assumptions include proportional hazards, linear form, and independence; violations can be addressed through stratification or time-dependent covariates.
\end{enumerate}

\begin{center}
\textbf{Questions for Further Discussion}
\end{center}

How should researchers address informative censoring and competing risks in survival analysis of HIV treatment cohorts in Africa? What role should time-dependent covariates play in capturing the dynamic nature of chronic disease progression and treatment effects?
\end{frame}
\end{document}
