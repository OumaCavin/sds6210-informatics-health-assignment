\documentclass[9pt,xcolor=dvipsnames,aspectratio=169]{beamer}
\usepackage[utf8]{inputenc}
\usepackage{amsmath,amssymb,graphicx,tikz,pgfplots,booktabs,siunitx}
\usetikzlibrary{arrows,shapes,decorations.pathmorphing,decorations.pathreplacing,decorations.snapping,fit,positioning,calc,intersections,shapes.geometric,backgrounds}
\usetheme[numbering=fraction,titleformat=smallcaps,sectionpage=progressbar]{metropolis}
\usepackage[style=authoryear]{biblatex}
\addbibresource{references.bib}
\setbeamertemplate{bibliography item}[text]
\graphicspath{{../assets/}}
\DeclareMathOperator{\e}{e}
\title{\Large SDS6210: Informatics for Health\\[0.3em]\small Advanced Quantitative: Maximum Likelihood Estimation and Logistic Regression}
\author{\textbf{Cavin Otieno}}
\institute{MSc Public Health Data Science\\Department of Health Informatics}
\date{\today}
\begin{document}
\begin{frame}[noframenumbering,plain]
    \maketitleslide
\end{frame}
\section{Maximum Likelihood Estimation: Fundamentals}
\begin{frame}{Definition: Maximum Likelihood Estimation}
Maximum Likelihood Estimation (MLE) is a general statistical method for estimating unknown parameters of a probability distribution. The MLE finds the parameter values that maximize the probability (likelihood) of observing the actual sample data.

For a sample of independent observations $x_1, x_2, \ldots, x_n$ from a distribution with probability density function $f(x;\theta)$ parameterized by $\theta$, the likelihood function is:

\[
L(\theta) = \prod_{i=1}^{n} f(x_i;\theta)
\]

The log-likelihood is computationally preferred:

\[
\ell(\theta) = \log L(\theta) = \sum_{i=1}^{n} \log f(x_i;\theta)
\]

The MLE $\hat{\theta}$ is:

\[
\hat{\theta} = \arg\max_{\theta} \ell(\theta)
\]

MLE properties under regularity conditions:
\begin{center}
\scalebox{0.75}{
\begin{tabular}{@{}llp{5cm}@{}}
\toprule \textbf{Property} & \textbf{Statement} & \textbf{Implication} \\
\midrowcolor
Consistency | $\hat{\theta} \xrightarrow{p} \theta$ as $n \to \infty$ | Consistent with true parameter |
Asymptotic normality | $\sqrt{n}(\hat{\theta} - \theta) \xrightarrow{d} N(0, I^{-1}(\theta))$ | Normal approximation for inference |
Efficiency | Achieves Cramér-Rao lower bound | Best unbiased estimator asymptotically \\
\bottomrule
end{tabular}}
\end{center}
\end{frame}
\begin{frame}{Likelihood Function: Binary Outcome Example}
Consider $n$ independent binary observations $y_1, y_2, \ldots, y_n$ where $y_i \sim \text{Bernoulli}(p)$. The likelihood is:

\[
L(p) = \prod_{i=1}^{n} p^{y_i} (1-p)^{1-y_i}
\]

For a sample with $k$ successes and $n-k$ failures:

\[
L(p) = p^k (1-p)^{n-k}
\]

The log-likelihood:

\[
\ell(p) = k \log p + (n-k) \log (1-p)
\]

Differentiating with respect to $p$:

\[
\frac{d\ell}{dp} = \frac{k}{p} - \frac{n-k}{1-p}
\]

Setting to zero and solving:

\[
\frac{k}{p} = \frac{n-k}{1-p} \implies k(1-p) = p(n-k) \implies k - kp = pn - pk \implies k = pn \implies \hat{p} = \frac{k}{n}
\]

The MLE of the Bernoulli probability is the sample proportion, which is the intuitive and correct estimate.
\end{frame}
\begin{frame}{Score Function and Information Matrix}
The score function (first derivative of log-likelihood) provides the basis for estimation:

\[
U(\theta) = \frac{\partial \ell(\theta)}{\partial \theta}
\]

The Fisher information matrix quantifies information about parameters:

\[
I(\theta) = -\frac{\partial^2 \ell(\theta)}{\partial \theta \partial \theta^T} = E\left[\left(\frac{\partial \ell}{\partial \theta}\right)\left(\frac{\partial \ell}{\partial \theta}\right)^T\right]
\]

For scalar parameter $\theta$:

\[
\text{Var}(\hat{\theta}) \approx \frac{1}{I(\theta)} = \frac{1}{-E\left[\frac{\partial^2 \ell}{\partial \theta^2}\right]}
\]

Example: Bernoulli distribution
\[
I(p) = \frac{n}{p(1-p)}
\]
\[
\text{Var}(\hat{p}) \approx \frac{p(1-p)}{n}
\]

The standard error is estimated as:
\[
\widehat{\text{SE}}(\hat{p}) = \sqrt{\frac{\hat{p}(1-\hat{p})}{n}}
\]
\end{frame}
\section{MLE for Logistic Regression}
\begin{frame}{Likelihood Function for Logistic Regression}
For logistic regression with binary outcomes $y_i \in \{0,1\}$ and covariates $\mathbf{x}_i = (x_{i1}, \ldots, x_{ip})$:

The probability model:
\[
P(Y_i = 1|\mathbf{x}_i; \boldsymbol{\beta}) = p_i = \frac{\exp(\beta_0 + \beta_1 x_{i1} + \cdots + \beta_p x_{ip})}{1 + \exp(\beta_0 + \beta_1 x_{i1} + \cdots + \beta_p x_{ip})}
\]

The likelihood function:
\[
L(\boldsymbol{\beta}) = \prod_{i=1}^{n} p_i^{y_i} (1-p_i)^{1-y_i}
\]

The log-likelihood:
\[
\ell(\boldsymbol{\beta}) = \sum_{i=1}^{n} \left[y_i \log p_i + (1-y_i) \log(1-p_i)\right]
\]

In terms of the logit function:
\[
\ell(\boldsymbol{\beta}) = \sum_{i=1}^{n} \left[y_i (\mathbf{x}_i^T \boldsymbol{\beta}) - \log(1 + \exp(\mathbf{x}_i^T \boldsymbol{\beta}))\right]
\]
\end{frame}
\begin{frame}{Score Equation and Estimation}
The score function for logistic regression:

\[
U(\boldsymbol{\beta}) = \frac{\partial \ell}{\partial \boldsymbol{\beta}} = \sum_{i=1}^{n} (y_i - p_i) \mathbf{x}_i = \mathbf{0}
\]

The observed information matrix:

\[
I(\boldsymbol{\beta}) = -\frac{\partial^2 \ell}{\partial \boldsymbol{\beta} \partial \boldsymbol{\beta}^T} = \sum_{i=1}^{n} p_i (1-p_i) \mathbf{x}_i \mathbf{x}_i^T
\]

Estimation procedure (Newton-Raphson algorithm):
\[
\boldsymbol{\beta}^{(t+1)} = \boldsymbol{\beta}^{(t)} + I^{-1}(\boldsymbol{\beta}^{(t)}) U(\boldsymbol{\beta}^{(t)})
\]

Convergence criteria:
\begin{center}
\scalebox{0.75}{
\begin{tabular}{@{}llp{5cm}@{}}
\toprule \textbf{Criterion} & \textbf{Threshold} & \textbf{Interpretation} \\
\midrowcolor
Change in log-likelihood | $< 10^{-6}$ | Sufficiently stable |
Change in coefficients | $< 10^{-4}$ | Converged estimates |
Maximum iterations | 25-100 | Prevents non-convergence \\
\bottomrule
\end{tabular}}
\end{center}
\end{frame}
\begin{frame}{Example: Estimating Logistic Regression Coefficients}
Consider a study of factors associated with hospital readmission ($n=500$ patients):

Model:
\[
\log\left(\frac{P(\text{readmission})}{1-P(\text{readmission})}\right) = \beta_0 + \beta_1 \times \text{Age}_{10} + \beta_2 \times \text{PriorHosp}
\]

Observed data:
\begin{center}
\scalebox{0.75}{
\begin{tabular}{@{}llp{5cm}@{}}
\toprule \textbf{Covariate} & \textbf{Coefficient ($\hat{\beta}$)} & \textbf{Standard Error} \\
\midrowcolor
Intercept ($\beta_0$) | -2.85 | 0.35 |
Age (per 10 years, $\beta_1$) | 0.42 | 0.08 |
Prior hospitalization ($\beta_2$) | 0.65 | 0.22 \\
\bottomrule
\end{tabular}}
\end{center}

Likelihood:
\[
\ell(\boldsymbol{\beta}) = \sum_{i=1}^{500} \left[y_i (\beta_0 + \beta_1 \text{Age}_{10i} + \beta_2 \text{PriorHosp}_i) - \log(1 + \exp(\beta_0 + \beta_1 \text{Age}_{10i} + \beta_2 \text{PriorHosp}_i))\right]
\]

The MLE was obtained after 8 Newton-Raphson iterations with log-likelihood convergence to $10^{-8}$.
\end{frame}
\begin{frame}{Interpretation of Logistic Regression Coefficients}
The exponentiated coefficients have direct interpretation as adjusted odds ratios:

For continuous predictor (Age, per 10 years):
\[
\exp(\hat{\beta}_1) = \exp(0.42) = 1.52
\]

Interpretation:
\[
\text{OR} = \frac{\text{Odds of readmission at age } a+10}{\text{Odds of readmission at age } a} = 1.52
\]

Each 10-year increase in age is associated with a 52\% increase in the odds of readmission, adjusting for prior hospitalization.

For binary predictor (Prior hospitalization):
\[
\exp(\hat{\beta}_2) = \exp(0.65) = 1.92
\]

Interpretation:
\[
\text{OR} = \frac{\text{Odds of readmission with prior hospitalization}}{\text{Odss of readmission without prior hospitalization}} = 1.92
\]

Patients with a prior hospitalization have 1.92 times higher odds of readmission, adjusting for age.

95\% confidence intervals:
\[
\text{OR}_j \in \left[\exp\left(\hat{\beta}_j - 1.96 \times \widehat{\text{SE}}\right), \exp\left(\hat{\beta}_j + 1.96 \times \widehat{\text{SE}}\right)\right]
\]
\end{frame}
\section{Public Health Applications}
\begin{frame}{MLE in Public Health Research}
MLE provides the statistical foundation for most public health regression analyses:

\begin{center}
\scalebox{0.75}{
\begin{tabular}{@{}llp{5cm}@{}}
\toprule \textbf{Application} & \textbf{Model} & \textbf{Parameter of Interest} \\
\midrowcolor
Disease risk factors | Logistic regression | Adjusted odds ratios |
Survival analysis | Cox proportional hazards | Hazard ratios |
Count outcomes | Poisson/negative binomial regression | Incidence rate ratios |
Time series | ARIMA models | Lag effects, seasonality \\
\bottomrule
\end{tabular}}
\end{center}

Example: Identifying risk factors for diabetes in Kenya
\[
\log\left(\frac{P(\text{diabetes})}{1-P(\text{diabetes})}\right) = \beta_0 + \beta_1 \text{BMI} + \beta_2 \text{Age} + \beta_3 \text{PhysicalActivity} + \beta_4 \text{Income}
\]

MLE estimates from a sample of 5,000 adults:
\[
\begin{aligned}
\hat{\beta}_1 &= 0.12 \quad (\text{BMI: OR} = 1.13 \text{ per unit}) \\
\hat{\beta}_2 &= 0.05 \quad (\text{Age: OR} = 1.05 \text{ per year}) \\
\hat{\beta}_3 &= -0.35 \quad (\text{Physical activity: OR} = 0.70) \\
\hat{\beta}_4 &= -0.02 \quad (\text{Income: OR} = 0.98)
\end{aligned}
\]

Public health interpretation: Physical activity has the strongest protective effect, with active individuals having 30\% lower odds of diabetes.
\end{frame}
\begin{frame}{Model Building and Selection in MLE Framework}
Variable selection methods within the MLE framework:

\textbf{Likelihood Ratio Test}:
\[
\text{LR} = -2(\ell_{\text{reduced}} - \ell_{\text{full}}) \sim \chi^2_{df_2 - df_1}
\]

Tests whether additional variables significantly improve model fit.

\textbf{Akaike Information Criterion (AIC)}:
\[
\text{AIC} = -2\ell + 2p
\]
where $p$ is the number of parameters. Lower AIC indicates better fit with penalty for complexity.

\textbf{Bayesian Information Criterion (BIC)}:
\[
\text{BIC} = -2\ell + p \log n
\]
Stronger penalty for model complexity, particularly useful for large samples.

Comparison of nested models:
\begin{center}
\scalebox{0.75}{
\begin{tabular}{@{}llp{5cm}@{}}
\toprule \textbf{Model} & \textbf{Log-Likelihood} & \textbf{AIC} \\
\midrowcolor
Age only | -342.5 | 687.0 |
Age + BMI | -325.2 | 654.4 |
Age + BMI + Physical Activity | -310.8 | 625.6 |
Age + BMI + Physical Activity + Income | -308.5 | 623.0 \\
\bottomrule
\end{tabular}}
\end{center}

The model with Age, BMI, and Physical Activity is selected based on substantial AIC improvement from adding Physical Activity, with minimal improvement from adding Income.
\end{frame}
\begin{frame}{Goodness of Fit Assessment}
Assessing model fit in the MLE framework:

\textbf{Hosmer-Lemeshow Test}:
\[
\chi^2 = \sum_{g=1}^{G} \frac{(O_g - E_g)^2}{E_g(1 - \frac{E_g}{n_g})}
\]
where groups are formed by predicted probabilities. Non-significant result indicates adequate fit.

\textbf{Deviance}:
\[
D = -2(\ell_{\text{fitted}} - \ell_{\text{saturated}})
\]
Lower deviance indicates better fit. Compare to $\chi^2_{n-p}$ distribution.

\textbf{Pseudo R-squared}:
\[
R^2_{\text{McFadden}} = 1 - \frac{\ell_{\text{fitted}}}{\ell_{\text{null}}}
\]
Values of 0.2-0.4 indicate good fit for logistic regression.

Receiver Operating Characteristic (ROC) curve:
\[
\text{AUC} = \int_0^1 \text{TPR}(\text{FPR}) \, d(\text{FPR})
\]
AUC = 0.82 indicates excellent discriminative ability.
\end{frame}
\section{LMIC Context: Sub-Saharan Africa}
\begin{frame}{MLE in African Public Health Research}
MLE-based methods are increasingly used in African health research:

\begin{center}
\scalebox{0.75}{
\begin{tabular}{@{}llp{5cm}@{}}
\toprule \textbf{Application} & \textbf{Setting} & \textbf{Example} \\
\midrowcolor
Risk factor analysis | DHS surveys | Identifying predictors of child stunting \\
Treatment evaluation | Clinical trials | Artemisinin combination therapy efficacy \\
Survival analysis | Cohort studies | HIV treatment outcomes, mortality prediction \\
Spatial analysis | Disease mapping & Malaria risk factors across districts \\
\bottomrule
end{tabular}}
\end{center}

Example: Risk factors for malaria in under-5 children (Kenya DHS 2022)

\[
\log\left(\frac{P(\text{positive RDT})}{1-P(\text{positive RDT})}\right) = \beta_0 + \beta_1 \text{Age} + \beta_2 \text{NetUse} + \beta_3 \text{Education} + \beta_4 \text{Wealth}
\]

Results:
\[
\begin{aligned}
\hat{\beta}_1 &= -0.08 \quad (\text{OR} = 0.92 \text{ per 6 months age}) \\
\hat{\beta}_2 &= -0.65 \quad (\text{OR} = 0.52 \text{ for net users}) \\
\hat{\beta}_3 &= -0.12 \quad (\text{OR} = 0.89 \text{ per maternal education level}) \\
\hat{\beta}_4 &= -0.05 \quad (\text{OR} = 0.95 \text{ per wealth quintile})
\end{aligned}
\]

Interpretation: Bed net use provides the strongest protection, reducing odds of malaria by 48\%.
\end{frame}
\begin{frame}{Challenges and Solutions for MLE in LMICs}
Challenges for MLE-based analysis in resource-limited settings:

\begin{center}
\scalebox{0.75}{
\begin{tabular}{@{}llp{5cm}@{}}
\toprule \textbf{Challenge} & \textbf{Impact} & \textbf{Solution} \\
\midrowcolor
Sample size limitations | Unstable estimates, wide CIs | Multi-site collaboration, pooling data |
Missing data | Biased estimates | Multiple imputation, sensitivity analysis |
Complex sampling | Standard errors underestimated | Survey design-based analysis |
Computational resources | Limited software access | Open-source tools (R, Python) \\
\bottomrule
\end{tabular}}
\end{center}

Complex survey analysis in MLE framework:
\[
\ell_{\text{survey}}(\boldsymbol{\beta}) = \sum_{h=1}^{H} \sum_{i=1}^{n_h} w_{hi} \ell_i(\boldsymbol{\beta})
\]
where $w_{hi}$ are survey weights accounting for sampling design.

Software recommendations for LMIC researchers:
\begin{center}
\scalebox{0.75}{
\begin{tabular}{@{}llp{5cm}@{}}
\toprule \textbf{Software} & \textbf{Cost} & \textbf{Survey Analysis Capability} \\
\midrowcolor
R | Free | Comprehensive (survey package) |
Stata | Commercial | Built-in survey commands |
Python | Free | statsmodels with survey module |
\bottomrule
\end{tabular}}
\end{center}
\end{frame}
\section{Summary}
\begin{frame}{Key Takeaways}
\begin{enumerate}
1. MLE finds parameter values that maximize the probability of observing the sample data, providing statistically efficient and consistent estimates.
\end{enumerate}

\begin{enumerate}
\resume{enumerate}
2. The likelihood function $L(\theta) = \prod f(x_i;\theta)$ and log-likelihood $\ell(\theta) = \sum \log f(x_i;\theta)$ are central to MLE.
\end{enumerate}

\begin{enumerate}
\resume{enumerate}
3. Logistic regression coefficients are estimated by maximizing the Bernoulli log-likelihood, yielding adjusted odds ratios.
\end{enumerate}

\begin{enumerate}
\resume{enumerate}
4. The score function and Fisher information matrix provide the basis for estimation and inference.
\end{enumerate}

\begin{enumerate}
\resume{enumerate}
5. MLE-based methods are widely used in African public health research but require attention to sampling design and data quality.
\end{enumerate}

\begin{center}
\textbf{Questions for Further Discussion}
\end{center}

How should researchers balance the need for complex multivariable models against the interpretability requirements for public health decision-making? What considerations are important when applying MLE-based methods to data from complex survey designs in LMIC contexts?
\end{frame}
\end{document}
