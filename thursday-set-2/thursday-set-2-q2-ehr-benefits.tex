\documentclass[9pt,xcolor=dvipsnames,aspectratio=169]{beamer}
\usepackage[utf8]{inputenc}
\usepackage{amsmath,amssymb,graphicx,tikz,pgfplots,booktabs,siunitx}
\usetikzlibrary{arrows,shapes,decorations.pathmorphing,decorations.pathreplacing,decorations.snapping,fit,positioning,calc,intersections,shapes.geometric,backgrounds}
\usetheme[numbering=fraction,titleformat=smallcaps,sectionpage=progressbar]{metropolis}
\usepackage[style=authoryear]{biblatex}
\addbibresource{references.bib}
\setbeamertemplate{bibliography item}[text]
\graphicspath{{../assets/}}
\DeclareMathOperator{\e}{e}
\title{\Large SDS6210: Informatics for Health\\[0.3em]\small Thursday Set 2, Q2: Benefits of Electronic Health Records}
\author{\textbf{Cavin Otieno}}
\institute{MSc Public Health Data Science\\Department of Health Informatics}
\date{\today}
\begin{document}
\begin{frame}[noframenumbering,plain]
    \maketitleslide
\end{frame}
\section{Definition and Theoretical Framework}
\begin{frame}{Definition: Electronic Health Records}
An Electronic Health Record (EHR) is a digital version of a patient's medical history, maintained by the healthcare provider over time, and may include all of the key administrative clinical data relevant to that person's care under a particular provider. The term "EHR" is often used interchangeably with "Electronic Medical Record" (EMR), though technically EMR refers to the digital version of paper charts within a single practice while EHR implies interoperability across multiple providers and settings.

The Healthcare Information and Management Systems Society (HIMSS) defines EHR as:
\begin{center}
"A longitudinal electronic record of patient health information generated by one or more encounters in any care delivery setting. This record includes patient demographics, progress notes, problems, medications, vital signs, past medical history, immunizations, laboratory data, and radiology reports."
\end{center}

Key characteristics that distinguish EHRs from paper records include:
\begin{itemize}
\item \textbf{Structure}: Standardized data fields with coded terminologies
\end{itemize}

\begin{itemize}
\item \textbf{Persistence}: Lifetime record maintained across encounters and providers
\end{itemize}

\begin{itemize}
\item \textbf{Accessibility**: Real-time availability to authorized users across settings
\end{itemize}

\begin{itemize}
\item \textbf{Interoperability**: Ability to exchange information with other health IT systems
\end{itemize}
\end{frame}
\begin{frame}{Theoretical Framework: EHR Benefits Model}
The benefits of EHR adoption can be understood through multiple theoretical lenses:

\begin{center}
\begin{tikzpicture}[scale=0.85]
\node[draw,rectangle,fill=blue!20,minimum width=2.5cm,minimum height=1.2cm] (C) at (0,1.5) {Clinical\\Benefits};
\node[draw,rectangle,fill=green!20,minimum width=2.5cm,minimum height=1.2cm] (O) at (4,1.5) {Operational\\Benefits};
\node[draw,rectangle,fill=red!20,minimum width=2.5cm,minimum height=1.2cm] (R) at (8,1.5) {Research\\Benefits};
\node[draw,rectangle,fill=yellow!20,minimum width=2.5cm,minimum height=1.2cm] (P) at (4,-0.5) {Public\\Health Benefits};
\node[draw,ellipse,fill=purple!10,minimum width=11cm,minimum height=3cm] (I) at (4,0.5) {};
\draw[->,thick] (C) -- (O);
\draw[->,thick] (O) -- (R);
\draw[->,thick] (R) -- (P);
\draw[->,thick] (P) -- (2,0.5) -- (C);
\node[draw,rectangle,fill=gray!20,minimum width=10cm,minimum height=0.6cm] at (4,-1.8) {Benefits cascade: Direct care improvements enable operational efficiencies that support research and population health};
\end{tikzpicture}
\end{center}

The Technology Acceptance Model (TAM) suggests that perceived usefulness and perceived ease of use determine EHR adoption. The DeLone and McLean IS Success Model extends this to include system quality, information quality, service quality, user satisfaction, and net benefits.

Research evidence has documented specific EHR benefits, though implementation quality significantly moderates outcomes. A systematic review found that EHR adoption was associated with:
\begin{itemize}
\item 15\% reduction in medication errors
\end{itemize}

\begin{itemize}
\item 12\% improvement in preventive care compliance
\end{itemize}

\begin{itemize}
\item Variable impact on clinical efficiency depending on implementation approach
\end{itemize}
\end{frame}
\section{Clinical and Operational Benefits}
\begin{frame}{Clinical Benefits of EHR Implementation}
EHRs generate clinical benefits through multiple mechanisms:

\begin{center}
\scalebox{0.75}{
\begin{tabular}{@{}llp{5cm}@{}}
\toprule
\textbf{Benefit Category} & \textbf{Mechanism} & \textbf{Evidence} \\
\midrowcolor
Medication Safety & Drug-drug interaction checking, allergy alerts, dose calculation & 30-50\% reduction in adverse drug events in CPOE systems \\
Clinical Decision Support & Evidence-based guideline reminders, order sets, risk calculators & 15-25% improvement in preventive care screening rates \\
Care Coordination & Referral tracking, care plan sharing, discharge summaries & 20\% reduction in hospital readmissions in integrated systems \\
Patient Safety & Legible orders, allergy documentation, barcode medication admin & 50\% reduction in transcription errors \\
Chronic Disease Management & Disease registries, population health dashboards, remote monitoring & 10-15% improvement in diabetes control (HbA1c) \\
\bottomrule
\end{tabular}}
\end{center}

The Computerized Provider Order Entry (CPOE) component of EHRs addresses one of the most common sources of medical errors: medication prescribing. By replacing handwritten prescriptions with structured electronic orders with clinical decision support, CPOE systems have demonstrated substantial reductions in prescribing errors.

Clinical decision support (CDS) extends EHR capabilities by presenting contextually relevant information at the point of care. Effective CDS includes drug-allergy checks, drug-drug interaction alerts, preventive care reminders, and evidence-based order sets.
\end{frame}
\begin{frame}{Operational and Financial Benefits}
Beyond direct clinical improvements, EHRs generate operational and financial benefits:

\begin{center}
\scalebox{0.75}{
\begin{tabular}{@{}llp{5cm}@{}}
\toprule
\textbf{Benefit} & \textbf{Description} & \textbf{Impact} \\
\midrowcolor
Reduced Chart Pull Time & Electronic search vs. physical file retrieval & 40-60\% time savings for medical records \\
Transcription Cost Savings & Structured data entry vs. dictation-transcription & 30-50\% reduction in documentation costs \\
Claims Processing & Automated coding, electronic claims submission & 15-25\% reduction in claim denials \\
Revenue Cycle Management & Charge capture, billing accuracy, collections & 5-10\% improvement in net revenue \\
Space and Storage & Digital records vs. physical file rooms & 60-80\% reduction in storage requirements \\
\bottomrule
\end{tabular}}
\end{center}

The financial business case for EHR adoption typically considers:
\begin{itemize}
\item \textbf{Cost savings}: Reduced labor, supplies, and physical infrastructure
\end{itemize}

\begin{itemize}
\item \textbf{Revenue enhancement}: Improved coding accuracy, reduced claim denials, faster reimbursement cycles
\end{itemize}

\begin{itemize}
\item \textbf{Cost avoidance**: Reduced adverse events, malpractice risk, regulatory penalties
\end{itemize}

Studies suggest that ambulatory practices typically achieve positive return on investment within 2-5 years of implementation, though this varies significantly based on implementation approach and practice characteristics.
\end{frame}
\section{Research and Public Health Applications}
\begin{frame}{Research Applications of EHR Data}
EHR data creates unprecedented opportunities for clinical research and quality improvement:

\begin{center}
\begin{tikzpicture}[scale=0.8]
\node[draw,rectangle,fill=blue!20,minimum width=2.5cm,minimum height=1.2cm] (OBS) at (0,1) {Observational\\Studies};
\node[draw,rectangle,fill=green!20,minimum width=2.5cm,minimum height=1.2cm] (PRAG) at (4,1) {Pragmatic\\Trials};
\node[draw,rectangle,fill=red!20,minimum width=2.5cm,minimum height=1.2cm] (PHENO) at (8,1) {Phenotype\\Discovery};
\node[draw,rectangle,fill=yellow!20,minimum width=2.5cm,minimum height=1.2cm] (AE) at (4,-1.5) {Pharmacovigilance};
\node[draw,rectangle,fill=purple!20,minimum width=2.5cm,minimum height=1.2cm] (QI) at (8,-1.5) {Quality\\Improvement};
\draw[->,thick] (OBS) -- (PRAG);
\draw[->,thick] (PRAG) -- (PHENO);
\draw[->,thick] (PHENO) -- (AE);
\draw[->,thick] (PHENO) -- (QI);
\node[draw,ellipse,fill=gray!10,minimum width=11cm,minimum height=3.5cm] (I) at (4,-0.2) {};
\node[draw,rectangle,fill=orange!20,minimum width=2cm,minimum height=1cm] (EHR) at (0,-1.5) {EHR Data};
\draw[->,thick] (EHR) -- (OBS);
\end{tikzpicture}
\end{center}

Key research applications include:
\begin{itemize}
\item \textbf{Cohort identification**: Rapid enrollment for clinical trials using structured phenotype definitions
\end{itemize}

\begin{itemize}
\item \textbf{Comparative effectiveness}: Real-world comparison of treatments using observational data with appropriate analytical methods
\end{itemize}

\begin{itemize}
\item \textbf{Safety surveillance**: Monitoring for adverse events across large patient populations
\end{itemize}

\begin{itemize}
\item \textbf{Natural history studies**: Understanding disease progression patterns in diverse populations
\end{itemize}

The FDA's Sentinel System and PCORnet demonstrate how EHR data networks can support post-market drug safety surveillance and pragmatic research at national scale.
\end{frame}
\begin{frame}{Public Health Benefits and Surveillance Integration}
EHR integration with public health systems creates population health surveillance capabilities:

\begin{center}
\scalebox{0.75}{
\begin{tabular}{@{}llp{5cm}@{}}
\toprule
\textbf{Integration Type} & \textbf{Function} & \textbf{Example} \\
\midrowcolor
Electronic Case Reporting & Automated notifiable disease reporting &实时 transmission of suspected COVID-19 cases to public health authorities \\
Syndromic Surveillance & Chief complaint and diagnosis data streams & Emergency department syndrome clustering for outbreak detection \\
Immunization Registries & Vaccination history and reminder systems & Childhood immunization tracking across providers \\
Cancer Registries & Tumor registry data extraction & Automated case ascertainment from pathology reports \\
\bottomrule
\end{tabular}}
\end{center}

During the COVID-19 pandemic, EHR-based electronic case reporting (eCR) enabled:
\begin{itemize}
\item Rapid identification of suspected cases for public health follow-up
\end{itemize}

\begin{itemize}
\item Reduction in reporting delays from days to hours
\end{itemize}

\begin{itemize}
\item Complete case information including demographics, symptoms, and risk factors
\end{itemize}

\begin{itemize}
\item Population-level dashboards for resource allocation and outbreak tracking
\end{itemize}

The Association of Public Health Laboratories and CDC have promoted eCR adoption through the AART (Automated Analysis and Reporting Technology) platform, which enables bidirectional communication between EHRs and public health agencies.
\end{frame}
\section{LMIC Context: Sub-Saharan Africa}
\begin{frame}{EHR Implementation in African Health Systems}
EHR implementation in Sub-Saharan Africa has evolved significantly, though coverage remains limited:

\begin{center}
\scalebox{0.7}{
\begin{tabular}{@{}llp{5cm}@{}}
\toprule
\textbf{System} & \textbf{Scope} & \textbf{Key Features} \\
\midrowcolor
OpenMRS & 100+ countries, 2M+ patients & Open-source, modular, customizable for LMIC contexts \\
AMPATH Medical Record System & Kenya, 200,000+ patients & HIV care focus, research integration, longitudinal follow-up \\
iqhealth & South Africa, national scale & Public sector integration, interoperability \\
CchiMed & Rwanda, national EMR & Integrated with national health insurance, primary care focus \\
\bottomrule
\end{tabular}}
\end{center}

Kenya's AMPATH (Academic Model Providing Access to Healthcare) system demonstrates EHR benefits in a resource-limited setting:
\begin{itemize}
\item HIV retention rates improved from 60\% to 85\% after EHR implementation
\end{itemize}

\begin{itemize}
\item Pharmacist intervention rates for drug interactions increased significantly
\end{itemize}

\begin{itemize}
\item Research productivity increased with standardized data for observational studies
\end{itemize}

However, significant challenges persist:
\begin{itemize}
\item Only 10-15\% of African health facilities have functional EHR systems
\end{itemize}

\begin{itemize}
\item Interoperability between vertical disease programs remains limited
\end{itemize}

\begin{itemize}
\item Sustainability depends on donor funding in many settings
\end{itemize}
\end{frame}
\begin{frame}{Case Study: OpenMRS Implementation for HIV Care in Kenya}
The AMPATH program implemented OpenMRS across 60+ facilities in western Kenya:

\begin{center}
\begin{tikzpicture}[scale=0.8]
\node[draw,rectangle,fill=blue!20,minimum width=2.5cm,minimum height=1.2cm] (R) at (0,1) {Registration\\Module};
\node[draw,rectangle,fill=green!20,minimum width=2.5cm,minimum height=1.2cm] (E) at (3.5,1) {Encounter\\Module};
\node[draw,rectangle,fill=red!20,minimum width=2.5cm,minimum height=1.2cm] (D) at (7,1) {Decision\\Support};
\node[draw,rectangle,fill=yellow!20,minimum width=2.5cm,minimum height=1.2cm] (RPT) at (3.5,-1) {Reporting\\Dashboard};
\node[draw,ellipse,fill=purple!10,minimum width=10cm,minimum height=2.5cm] (I) at (3.5,0) {};
\draw[->,thick] (R) -- (E);
\draw[->,thick] (E) -- (D);
\draw[->,thick] (E) -- (RPT);
\draw[->,thick] (D) -- (RPT);
\end{tikzpicture}
\end{center}

Key benefits observed:
\begin{itemize}
\item \textbf{Clinical}: 95\% retention in care at 12 months (vs. 70\% at paper-based sites)
\end{itemize}

\begin{itemize}
\item \textbf{Operational}: 50\% reduction in time for monthly reporting
\end{itemize}

\begin{itemize}
\item \textbf{Research**: >200 peer-reviewed publications using standardized EHR data
\end{itemize}

The program demonstrated that open-source EHR systems can achieve meaningful clinical benefits in LMIC settings when supported by adequate training and ongoing technical support.
\end{frame}
\section{Summary}
\begin{frame}{Key Takeaways}
\begin{enumerate}
1. EHRs provide clinical benefits through medication safety features, clinical decision support, and improved care coordination, with documented reductions in adverse events and improvements in preventive care.
\end{enumerate}

\begin{enumerate}
\resume{enumerate}
2. Operational benefits include reduced documentation time, improved billing accuracy, and decreased physical storage requirements, with most practices achieving positive ROI within 2-5 years.
\end{enumerate}

\begin{enumerate}
\resume{enumerate}
3. EHR data enables research applications including observational studies, pragmatic trials, pharmacovigilance, and quality improvement initiatives at unprecedented scale.
\end{enumerate}

\begin{enumerate}
\resume{enumerate}
4. Public health integration enables real-time surveillance, electronic case reporting, and population health monitoring that strengthen disease control programs.
\end{enumerate}

\begin{enumerate}
\resume{enumerate}
5. In African contexts, open-source systems like OpenMRS have demonstrated significant clinical and operational benefits, though coverage remains limited and sustainability challenges persist.
\end{enumerate}

\begin{center}
\textbf{Questions for Further Discussion}
\end{center}

How can African health systems overcome the barriers to EHR adoption while ensuring that implementations are sustainable and locally owned? What role should regional coordination play in developing interoperable health information systems across African countries?
\end{frame}
\end{document}
