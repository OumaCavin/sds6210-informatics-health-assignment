\documentclass[9pt,xcolor=dvipsnames,aspectratio=169]{beamer}
\usepackage[utf8]{inputenc}
\usepackage{amsmath,amssymb,graphicx,tikz,pgfplots,booktabs,siunitx}
\usetikzlibrary{arrows,shapes,decorations.pathmorphing,decorations.pathreplacing,decorations.snapping,fit,positioning,calc,intersections,shapes.geometric,backgrounds}
\usetheme[numbering=fraction,titleformat=smallcaps,sectionpage=progressbar]{metropolis}
\usepackage[style=authoryear]{biblatex}
\addbibresource{references.bib}
\setbeamertemplate{bibliography item}[text]
\graphicspath{{../assets/}}
\DeclareMathOperator{\e}{e}
\title{\Large SDS6210: Informatics for Health\\[0.3em]\small Tuesday Set 1, Q4: Types of Electronic Health Records}
\author{\textbf{Cavin Otieno}}
\institute{MSc Public Health Data Science\\Department of Health Informatics}
\date{\today}
\begin{document}
\begin{frame}[noframenumbering,plain]
    \maketitleslide
\end{frame}
\section{EHR, EMR, and PHR Definitions}
\begin{frame}{Definitions: EHR, EMR, and PHR}
These three related but distinct concepts represent different approaches to electronic health information management:

\textbf{Electronic Health Record (EHR)}:
"A longitudinal electronic record of patient health information generated by one or more encounters in any care delivery setting. This record includes patient demographics, progress notes, problems, medications, vital signs, past medical history, immunizations, laboratory data, and radiology reports." - HIMSS

\textbf{Electronic Medical Record (EMR)}:
"The electronic version of a patient's medical history, maintained by the healthcare provider over time, and may include all of the key administrative clinical data relevant to that person's care under a particular provider." - CDC

\textbf{Personal Health Record (PHR)}:
"An electronic application through which individuals can access, manage, and share their health information, and that of others for whom they are authorized, in a private, secure, and confidential environment." - HITECH Act

Key distinctions:
\begin{center}
\scalebox[0.75]{
\begin{tabular}{@{}llll@{}}
\toprule
\textbf{Feature} & \textbf{EHR} & \textbf{EMR} & \textbf{PHR} \\
\midrowcolor
Scope & Inter-organizational & Single organization & Patient-controlled \\
Data Source & Multiple providers & Single provider & Patient + providers \\
Owner & Healthcare organization & Healthcare organization & Patient \\
Portability & Limited (via HIE) & Low & High \\
\bottomrule
\end{tabular}}
\end{center}
\end{frame}
\begin{frame}{Conceptual Relationship Between EHR, EMR, and PHR}
\begin{center}
\begin{tikzpicture}[scale=0.85]
\node[draw,rectangle,rounded corners,fill=blue!20,minimum width=3.5cm,minimum height=2cm] (EMR) at (0,0) {\textbf{Electronic Medical Record}\\\small{Single organization focus}\\\small{Clinical documentation}};
\node[draw,rectangle,rounded corners,fill=green!20,minimum width=3.5cm,minimum height=2cm] (EHR) at (5,1) {\textbf{Electronic Health Record}\\\small{Interoperable across providers}\\\small{Complete patient history}};
\node[draw,rectangle,rounded corners,fill=red!20,minimum width=3.5cm,minimum height=2cm] (PHR) at (5,-1.5) {\textbf{Personal Health Record}\\\small{Patient-controlled}\\\small{Includes patient-generated data}};
\draw[->,thick,blue] (EMR) -- (EHR) node[midway,above] {Expand\\\small{Interoperability}};
\draw[->,thick,red] (EMR) -- (PHR) node[midway,below] {Add\\\small{Patient control}};
\draw[<->,thick,green] (EHR) -- (PHR) node[midway,right] {Data sync\\\small{Via APIs/HIE}};
\node[draw,ellipse,fill=purple!10,minimum width=11cm,minimum height=4.5cm] (Context) at (2.5,-0.2) {};
\node at (2.5,2) {\textbf{Evolving Health Information Landscape}};
\end{tikzpicture}
\end{center}

The terms EHR and EMR are often used interchangeably in practice, though technically EHR implies broader interoperability. The PHR represents a different paradigm where patients serve as the central coordinators of their own health information across multiple providers and sources.
\end{frame}
\section{Standalone vs Integrated Systems}
\begin{frame}{Standalone Electronic Health Record Systems}
A standalone EHR system operates independently of other healthcare information systems, maintaining its own database and user interface without direct connections to external systems.

Characteristics of standalone systems:
\begin{center}
\scalebox[0.75]{
\begin{tabular}{@{}llp{5cm}@{}}
\toprule
\textbf{Characteristic} & \textbf{Description} & \textbf{Implication} \\
\midrowcolor
Data Isolation & Data stored locally, no external connections & No automatic information sharing \\
Independent Operation & Works without network connectivity & Suitable for remote/fixed facilities \\
Limited Integration & No connection to labs, pharmacies, imaging & Manual processes for data exchange \\
Simpler Implementation & Fewer dependencies, less complexity & Faster deployment, lower initial cost \\
\bottomrule
\end{tabular}}
\end{center}

Advantages of standalone systems:
\begin{itemize}
\item Easier implementation and maintenance
\end{itemize}

\begin{itemize}
\item Lower infrastructure requirements
\end{itemize}

\begin{itemize}
\item Reduced security attack surface
\end{itemize}

\begin{itemize}
\item Useful in settings with limited connectivity

Disadvantages of standalone systems:
\begin{itemize}
\item Cannot access patient records from other providers
\end{itemize}

\begin{itemize}
\item Manual processes for referrals and care coordination
\end{itemize}

\begin{itemize}
\item Potential for incomplete patient information
\end{itemize}

\begin{itemize}
\item Difficulty in comprehensive population health analysis
\end{itemize}
\end{frame}
\begin{frame}{Integrated Electronic Health Record Systems}
Integrated EHR systems connect with multiple external systems to create a comprehensive view of patient information and enable seamless information exchange.

Components of integration:
\begin{center}
\scalebox[0.75]{
\begin{tabular}{@{}llp{5cm}@{}}
\toprule
\textbf{Integration Type} & \textbf{Connected System} & \textbf{Benefit} \\
\midrowcolor
Laboratory Information System & Lab results flow to EHR automatically & Faster results availability, reduced errors \\
Pharmacy System | Medication lists, drug interaction checking | Safety, inventory management \\
Radiology/PACS | Imaging studies linked to patient record | Complete clinical picture |
Health Information Exchange | Records from other institutions | Care coordination across providers |
\bottomrule
\end{tabular}}
\end{center}

Integration standards enabling connectivity:
\begin{itemize}
\item \textbf{HL7 v2.x}: Legacy message-based integration for lab results, ADT feeds
\end{itemize}

\begin{itemize}
\item \textbf{HL7 FHIR}: Modern RESTful API-based integration
\end{itemize}

\begin{itemize}
\item \textbf{IHE profiles**: Standardized integration patterns (XDS for document sharing)
\end{itemize}

Integrated systems require:
\begin{itemize}
\item Robust IT infrastructure
\end{itemize}

\begin{itemize}
\item Interoperability standards adoption
\end{itemize}

\begin{itemize}
\item Governance frameworks for data sharing
\end{itemize}

\begin{itemize}
\item Ongoing system maintenance and updates
\end{itemize}
\end{frame}
\begin{frame}{Standalone vs Integrated: Comparative Analysis}
\begin{center}
\scalebox[0.7]{
\begin{tabular}{@{}lll@{}}
\toprule
\textbf{Dimension} & \textbf{Standalone} & \textbf{Integrated} \\
\midrowcolor
Data Completeness | Limited to single organization | Comprehensive across providers \\
Care Coordination | Manual, fragmented | Automated, seamless \\
Implementation Cost | Lower initial investment | Higher total cost of ownership \\
Technical Complexity | Simpler architecture | Complex interconnected systems \\
Connectivity Requirements | Minimal | Robust network required \\
Scalability | Facility-level | Network/regional level \\
User Experience | Self-contained | Requires workflow integration \\
\bottomrule
\end{tabular}}
\end{center}

The choice between standalone and integrated systems depends on:
\begin{itemize}
\item \textbf{Organizational context}: Single facility vs. multi-facility health system
\end{itemize}

\begin{itemize}
\item \textbf{Connectivity infrastructure**: Available bandwidth, reliability
\end{itemize}

\begin{itemize}
\item \textbf{Budget constraints**: Initial investment vs. long-term total cost
\end{itemize}

\begin{itemize}
\item \textbf{Strategic priorities**: Care coordination goals, population health ambitions
\end{itemize}
\end{frame}
\section{Examples from Real Health Systems}
\begin{frame}{OpenMRS: Open-Source EHR in LMICs}
OpenMRS (Open Medical Record System) is the most widely deployed open-source EHR in low- and middle-income countries, with implementations across Africa, Asia, and the Caribbean.

Implementation statistics:
\begin{center}
\scalebox[0.75]{
\begin{tabular}{@{}ll@{}}
\toprule
\textbf{Metric} & \textbf{Value} \\
\midrowcolor
Countries with active implementations & 70+ \\
Total patients tracked | Over 20 million |
Primary use case | HIV/AIDS care, primary care |
Platform | Web-based, Java/MySQL |
\bottomrule
\end{tabular}}
\end{center}

Notable implementations:
\begin{itemize}
\item \textbf{AMPATH (Kenya)**: Over 500,000 patients, research integration, longitudinal follow-up
\end{itemize}

\begin{itemize}
\item \textbf{Rwanda national system**: Nationwide deployment in public sector facilities
\end{itemize}

\begin{itemize}
\item \textbf{Indonesia}: Adaptations for maternal and child health programs
\end{itemize}

Strengths of OpenMRS:
\begin{itemize}
\item Open-source with active global community
\end{itemize}

\begin{itemize}
\item Modular architecture for customization
\end{itemize}

\begin{itemize}
\item Support for offline operation in low-connectivity settings
\end{itemize}

\begin{itemize}
\item Extensive documentation and training resources
\end{itemize}
\end{frame}
\begin{frame}{Epic and Cerner: Commercial EHRs in High-Income Countries}
Epic Systems and Cerner (now Oracle Health) dominate the enterprise EHR market in the United States and increasingly in other high-income countries.

Epic Systems:
\begin{center}
\scalebox[0.75]{
\begin{tabular}{@{}ll@{}}
\toprule
\textbf{Metric} & \textbf{Value} \\
\midrowcolor
US Market Share | ~32\% of acute care hospitals |
Notable Users | Mayo Clinic, Cleveland Clinic, Kaiser Permanente |
Key Features | MyChart patient portal, Care Everywhere HIE |
Implementation Model | Large-scale enterprise deployment |
\bottomrule
\end{tabular}}
\end{center}

Cerner (Oracle Health):
\begin{center}
\scalebox[0.75]{
\begin{tabular}{@{}ll@{}}
\toprule \textbf{Metric} & \textbf{Value} \\
\midrowcolor US Market Share | ~24\% of acute care hospitals |
Notable Users | Veterans Health Administration, Sutter Health |
Key Features | Millennium platform, HealtheIntent population health |
Implementation Model | Enterprise with strong analytics capabilities |
\bottomrule
</tabular}
\end{center}

Both systems offer:
\begin{itemize}
\item Comprehensive clinical functionality
\end{itemize}

\begin{itemize}
\item Strong integration capabilities via FHIR
\end{itemize}

\begin{itemize}
\item Patient portal applications
\end{itemize}

\begin{itemize}
\item Population health management tools
\end{itemize}

Challenges include high implementation costs and significant workflow disruption during go-live.
\end{frame}
\begin{frame}{DHIS2: Aggregate Data System for Public Health}
While not a patient-level EHR, DHIS2 (District Health Information Software 2) is the most widely deployed health information system in low- and middle-income countries, serving as the national health management information system in over 70 countries.

DHIS2 characteristics:
\begin{center}
\scalebox[0.75]{
\begin{tabular}{@{}ll@{}}
\toprule \textbf{Metric} & \textbf{Value} \\
\midrowcolor Countries using DHIS2 | 70+ (mostly in Africa and Asia) |
Health facilities reporting | Over 100,000 facilities |
Primary data type | Aggregate service statistics |
Platform | Open-source, web-based |
\bottomrule
\end{tabular}}
\end{center}

DHIS2 functionality:
\begin{itemize}
\item Routine health facility reporting (service delivery, disease cases)
\end{itemize}

\begin{itemize}
\item Disease surveillance with threshold-based alerts
\end{itemize}

\begin{itemize}
\item Data quality validation and verification
\end{itemize}

\begin{itemize}
\item Dashboard visualization and report generation
\end{itemize}

\begin{itemize}
\item Program-specific modules (HIV, malaria, immunization)
\end{itemize}

Strengths include its open-source model, low cost, and adaptability to country-specific needs. Limitations include its focus on aggregate rather than patient-level data.
\end{frame}
\begin{frame}{Comparative Examples from Real Health Systems}
\begin{center}
\scalebox[0.65]{
\begin{tabular}{@{}llllll@{}}
\toprule \textbf{System} & \textbf{Type} & \textbf{Scope} & \textbf{Setting} & \textbf{Patients} & \textbf{LMIC Use} \\
\midrowcolor
OpenMRS | Patient-level EHR | Facility-based | All | 20M+ | Extensive |
AMPATH | Patient-level EHR | Research/academic | Kenya | 500,000+ | Extensive |
Cerner/Epic | Patient-level EHR | Enterprise | HIC | Millions | Limited |
i2b2 | Clinical data warehouse | Research | US/Europe | Cohorts | Growing \\
DHIS2 | Aggregate HMIS | National/regional | LMICs | N/A (population) | Dominant |
OpenMRS for Peds | Specialty EHR | Pediatric | Multiple | 100,000+ | Growing \\
\bottomrule
\end{tabular}}
\end{center}

Key observations:
\begin{itemize}
\item Patient-level EHRs (OpenMRS, Cerner, Epic) dominate high-resource settings
\end{itemize}

\begin{itemize}
\item Aggregate systems (DHIS2) serve routine reporting needs in LMICs
\end{itemize}

\begin{itemize}
\item Research platforms (i2b2, AMPATH) enable translational research
\end{itemize}

\begin{itemize}
\item No single system serves all purposes; integration between systems is essential
\end{itemize}

The trend toward interoperability (via FHIR) is enabling data exchange between previously siloed systems, moving toward comprehensive health information ecosystems.
\end{frame}
\section{LMIC Context: Sub-Saharan Africa}
\begin{frame}{EHR Adoption Patterns in Sub-Saharan Africa}
EHR adoption in Sub-Saharan Africa faces unique challenges and follows distinct patterns:

\begin{center}
\scalebox[0.75]{
\begin{tabular}{@{}llp{5cm}@{}}
\toprule \textbf{Challenge} & \textbf{Impact} & \textbf{Adaptation} \\
\midrowcolor
Limited Infrastructure | 40\% of facilities lack reliable electricity | Offline-capable systems, solar power |
Workforce Constraints | Shortage of IT staff, variable computer literacy | Simplified interfaces, training programs |
Fragmented Funding | Multiple donors, vertical programs | Integrated platforms, government ownership |
Paper Legacy | Historical data on paper | Parallel systems, selective digitization \\
\bottomrule
\end{tabular}}
\end{center}

Adoption strategies that work in African contexts:
\begin{itemize}
\item \textbf{Open-source systems**: Lower cost, local customization, community support
\end{itemize}

\begin{itemize}
\item \textbf{Offline-first design**: Functionality without continuous connectivity
\end{itemize}

\begin{itemize}
\item \textbf{Task-shifting**: Non-physician users (nurses, clerks) as primary operators
\end{itemize}

\begin{itemize}
\item \textbf{Incremental implementation**: Start with high-priority modules, expand gradually
\end{itemize}

\begin{itemize}
\item \textbf{Government ownership**: Integration with Ministry of Health priorities
\end{itemize}
\end{frame}
\begin{frame}{Case Study: EHR Implementation in Ghana}
Ghana's journey with EHR implementation illustrates both opportunities and challenges:

\textbf{EHR Systems in Ghana}:
\begin{center}
\scalebox[0.75]{
\begin{tabular}{@{}ll@{}}
\toprule \textbf{System} & \textbf{Scope} \\
\midrowcolor
MediTech + DHIS2 | Private hospital chains |
OpenMRS variants | Faith-based facilities, research sites |
Custom implementations | Teaching hospitals |
District Health Information System | National HMIS (aggregate data) |
\bottomrule
\end{tabular}}
\end{center}

Key lessons from Ghana's experience:
\begin{itemize}
\item \textbf{Integration matters**: Parallel systems without integration create inefficiencies
\end{itemize}

\begin{itemize}
\item \textbf{Training is essential}: User competency varies widely, ongoing training needed
\end{itemize}

\begin{itemize}
\item \textbf{Leadership drives adoption**: Hospital leadership commitment correlates with success
\end{itemize}

\begin{itemize}
\item \textbf{Patience is required**: Full implementation takes years, not months
\end{itemize}

The Ghana Health Service has prioritized EHR standardization and integration as part of its health sector transformation agenda.
\end{frame}
\section{Summary}
\begin{frame}{Key Takeaways}
\begin{enumerate}
1. EHR (inter-organizational), EMR (single-organization), and PHR (patient-controlled) represent different approaches to electronic health information.
\end{enumerate}

\begin{enumerate}
\resume{enumerate}
2. Standalone systems operate independently, while integrated systems connect with labs, pharmacies, imaging, and other providers.
\end{enumerate}

\begin{enumerate}
\resume{enumerate}
3. OpenMRS dominates patient-level EHR implementations in LMICs; DHIS2 serves aggregate reporting needs across 70+ countries.
\end{enumerate}

\begin{enumerate}
\resume{enumerate}
4. Commercial systems like Epic and Cerner serve enterprise markets in high-income countries.
\end{enumerate}

\begin{enumerate}
\resume{enumerate}
5. LMIC EHR adoption requires addressing infrastructure constraints, workforce limitations, and funding fragmentation through adapted approaches.
\end{enumerate}

\begin{center}
\textbf{Questions for Further Discussion}
\end{center}

What strategies can accelerate EHR adoption in African health facilities while ensuring sustainability and local ownership? How should the tension between standardization and customization be managed in diverse health system contexts?
\end{frame}
\end{document}
